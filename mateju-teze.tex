
\documentclass[FM,noheader,EN]{tulthesis}

% Thesis Statement
% Lukas Mateju
% ITE, FM TUL

% use XeLateX to compile

\newcommand{\verze}{1.4}
\usepackage{polyglossia}
\setmainlanguage[variant=american]{english}
\usepackage{fontspec}
\usepackage{xunicode}
\usepackage{xltxtra}
\usepackage{hyperref}
\usepackage{enumitem}
\usepackage{multirow}
\hypersetup{colorlinks=true, linkcolor=tulFM, urlcolor=tulFM, citecolor=tulFM}
\sloppy

\addto\captionsenglish{%
  \renewcommand{\bibname}{References}%
}

\TULtitle{Název tezí disertační práce}{A Modern Look on Speech Processing}
\TULprogramme{P2612}{Elektrotechnika a informatika}{Electrical Engineering and Informatics}
\TULbranch{2612V045}{Technická kybernetika}{Technical Cybernetics}
\TULauthor{Ing. Lukáš Matějů}
\TULsupervisor{Ing. Petr Červa, Ph.D.}
\TULyear{2017}
\TULthesisType{Teze disertační práce}{Thesis Statement}

\bibliographystyle{references} 
\titlespacing*{\chapter}{0pt}{-\topskip}{*4}

\begin{document}

\ThesisTitle{EN}
% \ThesisTitle{CZ}
\TULfooternopage
\nofootaddress


% poděkování
%\begin{acknowledgement}[wide]
%I'd like to thank myself etc. ...
%\end{acknowledgement}
%\clearpage


% abstrakty
\begin{abstractEN}[wide]
Abstract in English...

\vspace{0.5cm}
\noindent\textbf{Keywords:}
A, B, C, D, E
\end{abstractEN}

%\vspace{2cm}
\clearpage

%\begin{abstractCZ}[wide]
%Český abstrakt, protože proč by ne...

%\vspace{0.5cm}
%\noindent\textbf{Klíčová slova:}
%A, B, C, D, E
%\end{abstractCZ}
%\clearpage


% obsah
\TULfooter
\tableofcontents
\clearpage
%\listoffigures
%\listoftables


% seznam zkratek
\begin{abbrList}
\textbf{CNN} & Convolutional Neural Networks \\
\textbf{DNN} & Deep Neural Networks \\
\textbf{FAR} & False Alarm Rate \\
\textbf{FER} & Frame Error Rate \\
\textbf{HTER} & Half-Total Error Rate \\
\textbf{LID} & Language Identification \\
\textbf{MR} & Miss Rate \\
\textbf{SAD} & Speech Activity Detection \\
\textbf{SID} & Speaker Identification \\
\textbf{RNN} & Recurrent Neural Networks \\
\textbf{RTF} & Real-Time Factor \\
\textbf{VAD} & Voice Activity Detection \\
\textbf{WER} & Word Error Rate \\
\textbf{WFST} & Weighted Finite State Transducers \\
\end{abbrList}


% úvod
\chapter*{Introduction}
\addcontentsline{toc}{chapter}{Introduction}
Úvodní kapitola seznamující s obsahem tématu.
\begin{itemize}
	\item preprocessing řeči,
	\item detekce řeči/neřeči,
	\item detekce jazyka,
	\item změna mluvčího
	\item id mluvčího
	\item další segmentační úlohy, emoce
	\item -> rozpoznávání řeči s využitím získaných informací
\end{itemize}	

Seznámení s rozvržením práce a současným stavem: 
\begin{itemize}
	\item členění práce - jednotlivé kapitoly obsahují následující body
	\item SAD - hotovo,
	\item LID - rozpracováno,
	\item D+SID - další v pořadí,
	\item případné další úlohy (emoce atd.).
\end{itemize}	

Podobné systémy:
\begin{itemize}
	\item na čem založené,
	\item v čem se liší,
	\item proč znovu?
	\item porovnání.
	\item LIAM, Alisé
\end{itemize}	


% tělo
% rešeršní okénko
\chapter{An Overview of Field Research}
Rešerše pro jednotlivé bloky. 
Od vybraných původních metod po state-of-the-art.
Otázka... jak podrobně? - podle rozsahu ostatních kapitol?

\section{SAD}
state-of-the-art může mírně vycházet z úvodů článků, rozšíření
+ arpa challenges
\section{LID}
doba před ivectory, ivectory
\section{D+SID}


\chapter{Evaluation Metrics}
FER, MR, FAR, HTER, F-value, delta 2/3, RTF, rovnice


% použité technologie
\chapter{Proposed Solution}
\begin{itemize}
	\item představit na čem je založené...
	\item supervised machine learning, deep learning (torch, gpus), liší podle modulu
	\item podrobnější představení technologií? viz DNN apod.?
\end{itemize}

\section{Employed Technologies}
fbc, DNN, WFST (dekodér) (SAD)

fbc + jak bottleneck příznaky, ivectory a jak kombinovat s DNN (LID)

otázka? ... jak podrobně
\subsection{Deep Neural Networks}
\subsection{Weighted Finite State Transducers}
\subsection{iVectors}

\section{Experimental Setup}
styl použití technologií, idea, DNN na trénování, WFST vyhlazení atd.

základní nastavení, torch, náš rozpoznávač atd.


% návrh modulů
\chapter{Development of the Proposed Solution}
vývoj jednotlivých bloků, hlavní část tvoří SAD, zbytek LID.

\section{SAD}
\begin{itemize}
	\item vychází z publikovaných článků. SIGMAP, ICASSP a SPRINGER,
	\item od začátku návrhu až po finální modely WFST. Vyhodnocení na QUT-NOISE-TIMIT, rozšířené testy ze springeru,
	\item vliv SADu na rozpoznávač, 
	\item +/- vše publikované + pár dílčích věcí, co se nepoužilo.
	\item zdůraznění výsledků + že se používá na tul
\end{itemize}	

\section{LID}
\begin{itemize}
	\item bez ivectorů, fbc přínaky, bottleneck příznaky,
	\item ivectory úvod,
	\item (zatím nepublikováno)
\end{itemize}	

Možno 'vykrást' publikované články.


% závěr
\chapter{Conclusions}
Závěrečná kapitola shrnující celkový stav práce:
\begin{itemize}
	\item SAD (hotovo) + zhodnocení výsledků \cite{SIGMAP16} \cite{ICASSP17} \cite{SPRINGER17}, 
	\item LID (rozpracováno) + zhodnocení dosavadních výsledků + co dál,
	\item D+SID (bude) + co dál,
	\item případné další úlohy.
\end{itemize}	
Případné další směrování práce.

%This is a simple equation for a real-life threatening question: 
%\begin{equation}
%\label{eq_dp}
%	DP = \frac{D}{P} \enspace,
%\end{equation}
%where $D$ is Dany and $P$ is ...


% reference
\bibliography{references}
\addcontentsline{toc}{chapter}{References}


% seznam publikací
\newpage
\chapter*{Author's Publications}
\addcontentsline{toc}{chapter}{Author's Publications}
% PŘIDAT - doplnit stránky u TSD, doplnit stránky u SPRINGER (jestli vyjde?)
	
	\textbf{2017:}
	\begin{enumerate}[leftmargin=*]
		\bibitem{SPRINGER2017}
		L. Mateju, P. Cerva, and J. Zdansky, “Investigation into the Use of WFSTs and DNNs for 
		Speech Activity Detection in Broadcast Data Transcription,” in \textit{E-Business and
		Telecommunications - 13th International Joint Conference, ICETE 2016, Lisbon,
		Portugal, July 26-28, 2016, Revised Selected Papers}, pp. 1–18, 2017.
	
		\bibitem{TSD2017}
		R. Safarik and L. Mateju, “The Impact of Inaccurate Phonetic Annotations 
		on Speech Recognition Performance,” in \textit{Text, Speech, and Dialogue - 20th 
		International Conference, TSD 2017, Prague, Czech Republic, August 27-31, 2017, 
		Proceedings}, pp. 1–8, 2017.
	
		\bibitem{ICASSP2017}
		L. Mateju, P. Cerva, J. Zdansky, and J. Malek, “Speech Activity Detection in 
		Online Broadcast Transcription Using Deep Neural Networks and Weighted Finite
		State Transducers,” in \textit{2017 IEEE International Conference on Acoustics, Speech 
		and Signal Processing, ICASSP 2017, New Orleans, LA, USA, March 5-9, 2017}, 
		pp. 5460–5464, 2017.
	\end{enumerate}
	
	\noindent \textbf{2016:}
	\begin{enumerate}[leftmargin=*,resume]
		\bibitem{TSD2016}
		M. Bohac, L. Mateju, M. Rott, and R. Safarik, “Automatic Syllabification and 
		Syllable Timing of Automatically Recognized Speech - for Czech,” in \textit{Text, Speech, 
		and Dialogue - 19th International Conference, TSD 2016, Brno, Czech Republic, 
		September 12-16, 2016, Proceedings}, pp. 540–547, 2016.
	
		\bibitem{SIGMAP2016}
		L. Mateju, P. Cerva, and J. Zdansky, “Study on the Use of Deep Neural Networks 
		for Speech Activity Detection in Broadcast Recordings,” in \textit{Proceedings of the 13th 
		International Joint Conference on e-Business and Telecommunications (ICETE 
		2016) - Volume 5: SIGMAP, Lisbon, Portugal, July 26-28, 2016}, pp. 45–51, 
		2016.
		
		\bibitem{TSP2016}
		R. Safarik and L. Mateju, “Impact of Phonetic Annotation Precision on Automatic 
		Speech Recognition Systems,” in \textit{39th International Conference on 
		Telecommunications and Signal Processing, TSP 2016, Vienna, Austria, June 27-29, 2016}, 
		pp. 311–314, 2016.
	\end{enumerate}
	
	\noindent \textbf{2015:}
	\begin{enumerate}[leftmargin=*,resume]
		\bibitem{ECMSM2015}
		L. Mateju, P. Cerva, and J. Zdansky, “Investigation into the Use of Deep Neural 
		Networks for LVCSR of Czech,” in \textit{2015 IEEE International Workshop of Electronics, 
		Control, Measurement, Signals and their Application to Mechatronics, ECMSM, 
		2015, Liberec, Czech Republic, June 22-24, 2015}, pp. 184–187, 2015.
	\end{enumerate}
\end{document}
